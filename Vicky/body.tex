% Especificaciones del tamaño de letra, tamaño de hoja, márgenes, librerias, etc.
\documentclass[12pt, letterpaper]{article}
\usepackage[english]{babel}
\usepackage{fancyhdr}
\usepackage[utf8]{inputenc}
\usepackage[T1]{fontenc}
\usepackage{mathrsfs}
\usepackage{amsmath}
\usepackage{graphicx}
\usepackage{subcaption}
\usepackage[hidelinks]{hyperref}
\usepackage{url}
\usepackage{amssymb}
\usepackage{float}
\usepackage[framed, numbered]{matlab-prettifier}
%\usepackage[framed, numbered, autolinebreaks, useliterate]{mcode}
\usepackage[margin=1in]{geometry}
\renewcommand{\baselinestretch}{1.5}

% Enlace Bibliografía
\usepackage{csquotes}
\usepackage[notes,backend=biber]{biblatex-chicago}
\addbibresource{referencias.bib}

% Titulo, autores, fecha.
\title{Sobre "El Hombre en Busca de Sentido"}
\author{Victoria Rábago \and 1155268}
\date{Mayo 9, 2019}
\pagestyle{fancy}
\fancyhf{}
\rhead{Victoria Rábago}
\lhead{Sobre "El Hombre en Busca de Sentido"}
\rfoot{\thepage}
% Inicio del documento
\begin{document}
\maketitle
\pagenumbering{gobble}
\section*{Introducción}
Son innegables las atrocidades que han acontecido a lo largo de la historia. Éstas nos pueden enseñar mucho sobre nuestra verdadera naturaleza como seres humanos y aquellas personas que han analizado estos acontecimientos nos han acercado un paso más a comprender nuestra manera de sentir y ser. Uno de estos personajes fue el psiquiátra Viktor E. Frankl, quien, después de haber experimentado la crudeza de la Segunda Guerra Mundial y los campos de concentración, relata la manera en la que la mente de aquellos prisioneros cambió radicalmente la manera en la que se desenvolvían con su entorno. El análisis hecho por este personaje lo llevó a desarrollar la teoría de la logoterapía y estudiar cómo el individuo se enfrenta ante las adversidades de la vida y los mecánismos de defensa psicológicos que se desarrollan en él para sobrevivir.

En su escrito, \textit{"El Hombre en Busca de Sentido"}, se discuten distintas maneras en que las personas en situaciones muy desfavorables logran superar dificultades y cuáles son las consecuencias de las conductas que toman los reclusos dentro de los campos de concentración. Hay bastantes puntos interesantes que se realizan en su documento, sin embargo me gustaría hablar de tres ideas importantes hechas por Frankl, y expresar qué es aquello con lo que concuerdo y desacuerdo. Primeramente expondremos la manera en la que el autor argumenta que la actitud que \textit{decidamos} tomar ante la vida y nuestros problemas es mucho más importante que la situación por la que estamos pasando como tal. Seguido de este análisis veremos cómo, para evitar suicidios, Frankl hace hincapié en ver hacia el futuro y aferrarnos a aquello que está fuera de nuestra persona, obteniendo así la motivación necesaria para seguir resistiendo el sufrimiento que es inherente a la existencia. Por último compararemos los momentos que Frankl vivió después de su reclusión con las ideas descritas anteriormente.

En las siguientes secciones desarrollaremos más los cimientos que hemos establecido. Sin más preámbulos empecemos a examinar los distintos puntos realizados.

\newpage
\pagenumbering{arabic}
\section*{Desarrollo}
\subsection*{La libertad interior y la angustia}

Dentro de su obra, Frankl hace la pregunta \textit{"¿Es cierta la teoría que nos enseña que el hombre no es más que el producto de muchos factores condicionales, sean de naturaleza bilógica, psicológica o sociológica?"} Él hace esta pregunta con el objetivo de entender si los prisioneros de los campos de concentración, que actuaban de una manera tan anormal ante las atrocidades que vivían día a día, se comportaban de esta manera debido a los factores ambientales que estaban expuestos.

Estos condenados sufrían castigos horribles y sus vidas eran realmente miserables, sin embargo encontrabámos a muchos de ellos apáticos ante estas dificultades. Es natural preguntarse si el presidiario no puede escapar a esta conducta debido a los factores externos con los que se enfrenta, o si éste es realmente libre de elegir la manera en la que se comporta.

La conclusión a la que llegó Frankl fue que el hombre \textit{tiene la capacidad de elección}. La experiencia que vivió en los campos le demostró que a pesar de lo que vivían estos desdichados, había individuos que eran capaces de consolar a los demás y dar su último trozo de pan a alguien debastado. Estos ejemplos, aunque escazos, probaron para él que el hombre puede conservar su independencia mental en situaciones donde todo conspire en su contra. Se refirió a esto como "\textit{la última de las libertades humanas ---la elección de la actitud personal ante un conjunto de circunstancias--- para decidir su propio camino.}"\autocite{frankl91}

Esa libertad espiritual es aquella que nos propociona el propósito y sentido a la vida. De manera análoga, Jean-Paul Sartre ya ha hablado un poco sobre esto dentro de la perspectiva existencialista. En su colección de cátedras titulada "\textit{El existencialismo es un humanismo}", Sartre habla sobre cómo el ser humano es totalmente libre de tomar cualquier decisión, y sea cual sea la decisión que tome, ésta será la correcta siempre y cuando lo sea para el individuo que la tomó. Nadie más puede decidir por otro individuo y esta libertad abrumadora llena a la gente de un sentimiento que él denominó "\textit{angustia}".

Esta angustia puede llevarnos a evitar la toma de decisiones por completo, sin embargo estaríamos negando nuestra libertad absoluta, comportándonos así como un objeto inerte a la deriva de las decisiones de otros. A esto Sartre lo denominó como "\textit{mala fe}", a estar a merced de eventos impredecibles incapaz de tomar decisiones. Dadas estas circunstancias, nosotros somos responsables de la actitud que tenemos ante la vida y con esto también se nos proporciona la responsabilidad de darle un propósito a nuestra vida. Al hacernos responsables de nosotros y de nuestro camino se nos brinda una oportunidad de entender nuestros méritos, y así poder lidiar más fácilmente con los problemas que se nos cruzan en el camino.

Un gran problema ante situaciones que parecen totalmente perdidas (dígase por ejemplo el condenado que no sabe el tiempo que duraría dentro del campo de concentración, un paciente con un diagnóstico terminal, etcétera) es la falta de motivación que se presenta. Una situación así drena la energía emocional de cualquier individuo, pero cómo nosotros elijamos actuar ante esas situaciones puede ser la diferencia entre seguir siendo libres o caer en mala fe. Aquellas personas que han demostrado tener esa libertad espiritual por lo regular tienen muy claro algo que los impulsa a buscar ese porvenir. El futuro es algo prometedor para ellos y se aferran a esto para seguir adelante.


\subsection*{El porvenir interno y externo}

Los individuos que saben aquello que les apasiona y se esfuerzan por conseguir esas metas que tienen en mente son los que regularmente superan las adversidades que aparecen en su vida. Esto le quedaba bastante claro a Frankl, quien utilizaba este conocimiento para lograr animar a sus camaradas a seguir adelante luchando contra la calidad de vida que llevaban en los campos. En la seccion dos de su obra, existe una subsección llamada \textit{Algo nos espera}, en donde describe cómo intervino para evitar que dos hombres se suicidaran. Los hombres argumentaban que ya no esperaban nada de la vida, sin embargo se les presenta un contraargumento con el cual los hacían comprender que la vida aún esperaba algo de ellos. Se les hizo entrar en razón diciéndoles que había seres queridos esperándolos allá afuera y había obras en progreso que sólo ellos podían terminar. Esta promesa del porvenir que los aguardaba allá fuera los hacía reconsiderar sus acciones y con estas metas en mente sus pensamientos se tranquilizaban.

La idea de un porvenir externo para motivar a otras personas puede ser muy útil en muchas ocasiones, es bastante cierto que somos irremplazables y que quizá sólo nosotros podamos continuar con aquellos proyectos y relaciones que hemos venido construyendo. Sin embargo para un contexto más amplio quizá no sea la mejor de las decisiones basar toda la motivación personal de un individuo en factores externos a él.

Lo que sugiero es formar la idea de un \textit{porvenir interno}, aquello que nosotros esperamos de nosotros mismos y que esté basado en nuestra introspección. Hay una cita muy famosa de Orson Welles que dice "\textit{Nacemos solos, vivimos solos y morimos solos. Sólo a través de nuestro amor y amistad podemos crear la lilusión de no estar solos.}" Desglosando un poco la idea que transmite esta cita, podemos interpretarlo como el hecho de que es necesario conocernos y tener una relación sana con nosotros mismos. La realidad es que no siempre nos es posible contar con otros seres queridos, pero siempre nos tendremos a nosotros mismos, por lo que basarnos en nuestro propio ser para formar motivaciones lo considero mucho más efectivo.

Esto no quiere decir que aquello que uno mismo busca fuera de su ser esté mal, sin embargo es más propenso a verse afectado por terceros y terminar siendo falsas ilusiones, o incluso metas inalcanzables. Es comprensible que en situaciones tan desesperadas como un campo de concentración las personas afectadas se aferren a ese futuro que hay allá afuera y que depende del mundo exterior, sin embargo siempre será preferible buscar un futuro que dependa de uno mismo, ya que si las expectativas del sujeto no se cumplen por lo regular esto ocasiona decepción y una pérdida de motivación, lo cual nos regresaría al estado inicial de apatía por la vida. 

\subsection*{Decepción}

Sea cual sea la naturaleza del porvenir que el sujeto tenga arraigada a su ser, éste le dará la fuerza de voluntad para actuar y salir de su mala fe. Se mantendrá mentalmente independiente y podrá con las adversidades que se le presentan. Esto ya por sí mismo es efectivo en el caso de tener como objetivo final hacer que el individuo salga de la situación desfavorable, pero algo que no se ha analizado es el efecto que tendrán expectativas que no se cumplan al momento de superar la dificultad experimentada.

Dentro de la obra de Frankl, en los últimos pasajes del texto encontramos relatos sobre cómo algunos ex reclusos que finalmente salían del infierno que experimentaron durante años podían finalmente cumplir con esas metas que tanto habían añorado. Lamentablemente cuando éstos volvían a sus ciudades de origen encontraban escenarios totalmente distintos a los que ellos imaginaban. La meta que tanto los mantenía a flote se desvanecía de distintas maneras. Algunos encontraron que nadie los esperaba de vuelta, indiferencia hacia ellos y su situación. ¿Y qué es lo que se podía esperar? El mundo avanzó durante años sin ellos. La crueldad de esa realidad es algo que ningún hombre o niño debería de sufrir.

La decepción que sufren estos individuos fue inevitable al momento de vertir sus esperanzas y metas en todo aquello fuera de su control. Este es uno de los motivos por los cuales considero más efectivo el utilizar motivaciones basadas en nosotros mismos. Esto no quiere decir que tener esperanzas en otros seres humanos sea completamente en vano, a pesar de la decepción que esto pueda traer. Es mucho más favorable en ciertas ocasiones tener un porvenir externo que uno interno, sin embargo considero que utilizar el interno siempre que sea posible será más favorable para un desarrollo personal y emocional. 

Exigirle a un individuo en esa situación aferrarse a algo más personal es mucho más difícil, a duras penas se aferran a algo siquiera. Pero si lo llevamos a un contexto más grande, a uno no tan radical, es un poco más razonable que una persona que sufra algún tipo de problema en su vida intente realizar introspección y así hallar aquello que realmente le apasiona, le motiva a seguir adelante. Es hasta este punto en el que esta persona logrará conseguir esa motivación que le ayudará a salir de los problemas a los que se enfrenta, o por lo menos poder enfrentarlos con dignidad y la responsabilidad que conlleva.
\newpage

\section*{Conclusión}

Después de las secciones anteriores podemos observar cómo la filosofía de Frankl ayuda a aquellas personas apáticas ante la vida. Es  esa falta de motivación y de sentido que el ser humano desprecia y no sabe cómo deshacerse de ese sentimiento. Lo que queda para poder combatirlo es tomar la responsabilidad de hacer algo al respecto, salir de la mala fe en la que nos encontramos y plantearnos objetivos nosotros mismos. Es con esto que nos podemos sacudir el sentimiento de amargura que trae la falta de dirección en la vida. 

Pensar en el porvenir es una manera efectiva de mantenernos firmes en nuestro propósito de salir de las adversidades planteadas sobre nuestro camino,  enfrentándolas y asumiendo con responsabilidad las consecuencias de estas dificultades.

A pesar de la efectividad de visualizar un futuro, he encontrado más favorable aquél en el que depende meramente de un mismo individuo, esto con el fin de desarrollarnos mucho mejor personalmente y evitar decepciones innecesarias. 

Puede que el razonamiento detrás de todo este ensayo no sea el más adecuado. Quizá la educación que he recibido o la edad que tengo no me permitan ver el panorama completo pero por ahora esta es a la conclusión que he llegado.
\newpage
\pagenumbering{gobble}
%%%%%  Bib
\renewcommand\refname{Referencias}
\printbibliography
\end{document}
