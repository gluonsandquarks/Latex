% Especificaciones del tamaño de letra, tamaño de hoja, márgenes, librerias, etc.
\documentclass[12pt, letterpaper]{article}
\usepackage[english]{babel}
\usepackage{fancyhdr}
\usepackage[utf8]{inputenc}
\usepackage[T1]{fontenc}
\usepackage{amsmath}
\usepackage{graphicx}
\usepackage{subcaption}
\usepackage[hidelinks]{hyperref}
\usepackage{url}
\usepackage{amssymb}
\usepackage{float}
\usepackage[margin=1in]{geometry}
\usepackage{listings}
\usepackage{verbatim}
\input{arduinoLanguage.tex}
\renewcommand{\baselinestretch}{1.5}

% Enlace Bibliografía
\usepackage{csquotes}
\usepackage[notes,backend=biber]{biblatex-chicago}
\addbibresource{referencias.bib}

% Titulo, autores, fecha.
\title{Práctica \#2: Simulación de un Semáforo}
\author{Carlos A. Vásquez Castañeda \and 1155057 \and Grupo 395}
\date{Febrero 12, 2020}
\pagestyle{fancy}
\fancyhf{}
\rhead{Instrumenación}
\lhead{Práctica \#2}
\rfoot{\thepage}


% Inicio del documento
\begin{document}
\maketitle
\section*{Introducción}

Los sistemas electrónicos con microcontroladores se utilizan en muchos dispositivos hoy en día. Comprenderlos resulta de gran ventaja y realizar un programa que simule uno de estos dispositivos resulta en un beneficio para la comprensión de estos sistemas.

\section*{Objetivo}

El objetivo de esta práctica es reforzar aquello enseñado por la práctica \#1 y volverlo un poco más complicado para entender la lógica de programación detrás de Arduino. 

\section*{Marco Teórico}
Para la elaboración de el semáforo es necesario entender algunas funciones básicas:

\begin{itemize}
	\item pinMode(pin, mode): configura el modo de salida de cada conexión disponible en el microcontrolador de Arduino.
	\item digitalWrite(pin, HIGH/LOW): escribe un valor de ALTO o BAJO en la conexión del contacto que se ha elegido, sean 5 volts o 3.3 volts.
\end{itemize}

\section*{Materiales y Equipo}
\begin{itemize}
	\item Arduino UNO
	\item Arduino IDE
	\item Proteus 8 Professional
	\item LEDs (3)
	\item Resistencia 220 $\Omega$ (3)
	\item Referencias de internet (StackExchange, Arduino.cc)
\end{itemize}

\section*{Desarrollo}
Dada la simulación que deseamos realizar es la de un semáforo, tendremos que configurar 3 luces distintas para que actúen como aquellas que tienen un semáforo de verdad. Para esto utilizaremos tres LEDs, uno verde, uno amarillo y otro rojo. Estos los definimos al inicio del código del programa, como se muestra a continuación.

\begin{lstlisting}[language=Arduino]
	int v = 13;
	int a = 12;
	int r = 11;
\end{lstlisting}

Una vez asignadas estas variables a los distintas conexiones del microcontrolador, es necesario definirlas como posibles salidas o entradas. Dado el funcionamiento que queremos obtener del controlador, será necesario configurar estas conexiones como salidas.

\begin{lstlisting}[language=Arduino]
	void setup () {
		pinMode(v, OUTPUT);
		pinMode(a, OUTPUT);
		pinMode(r, OUTPUT);
		}
\end{lstlisting}

Finalmente sigue regular el encendido y apagado de las distintas luces que estarán a nuestra disposición. Para esto, considerando que aún no hemos aprendido estructuras de código complejas, utilizaremos lo que hemos aprendido en la práctica \#1 para realizar el siguiente bloque de código.

\begin{lstlisting}[language=Arduino]
	void loop () {
		digitalWrite(v, 1);
		delay(5000);

		digitalWrite(v, 0);
		delay(500);
		digitalWrite(v, 1);
		delay(500);
		digitalWrite(v, 0);
		delay(500);
		digitalWrite(v, 1);
		delay(500);
		digitalWrite(v, 0);
		delay(500);
		digitalWrite(v, 1);
		delay(500);
		digitalWrite(v, 0);
		delay(500);
		digitalWrite(v, 1);
		delay(500);
		digitalWrite(v, 0);
		delay(500);
		digitalWrite(v, 1);
		delay(500);
		digitalWrite(v, 0);
	}
\end{lstlisting}

En el bloque de código anterior enciende el LED de color verde y espera durante 5000 milisegundos o 5 segundos para tomar la siguiente acción. Posteriormente procede a apagar el LED verde, esperar medio segundo, volverlo a encender y esperar otro medio segundo. Todo esto toma lugar cinco veces para simular el parpadeo de un semáforo convencional.

A continuación, inmediatamente después de la última vez que el LED verde se apaga, procederemos a encender el LED amarillo para que no exista ninguna inconsistencia y el semáforo funcione de manera adecuada.

\begin{lstlisting}[language=Arduino]
	digitalWrite(a, 1);
	delay(1000);
	digitalWrite(a, 0);
\end{lstlisting}

La función $delay$ hace esperar al controlador 1 segundo para tomar la siguiente acción que es apagar el LED amarillo, posteriormente se realiza el siguiente bloque de código para encender el LED rojo, dejarlo encendido durante 2 segundos y posteriormente apagarlo.

\begin{lstlisting}[language=Arduino]
	digitalWrite(r, 1);
	delay(2000);
	digitalWrite(r, 0);
\end{lstlisting}

Y dado el bucle infinito, este programa actuará de esta manera las veces que sean necesarias hasta que se desenergize el circuito.

\subsection*{Diagrama de Flujo}
Observando el proceso mediante bloques de diagrama de flujo luciría similar a esto:

\begin{figure}[H]
	\centering
	\includegraphics[width=\textwidth]{flowchart.png}
	\caption{Proceso del programa elaborado anteriormente.}
\end{figure}

Como es posible observar, la complejidad del programa ha aumentado un poco, sin embargo sigue siendo bastante lineal y el microcontrolador no toma decisiones en base a otros datos. Podríamos decir que sólo sigue una serie de instrucciones indefinidamente y es por eso que nuestro diagrama de flujo se muestra tan simple.

\subsection*{Código}
A continuaciön se muestra el código en su totalidad.

\begin{lstlisting}[language=Arduino]
int v = 13;
int a = 12;
int r = 11;

void setup () {
  pinMode(v, OUTPUT);
  pinMode(a, OUTPUT);
  pinMode(r, OUTPUT);
}


void loop () {
  
  digitalWrite(v, 1);
  delay(5000);
  
  digitalWrite(v, 0);
  delay(500);
  digitalWrite(v, 1);
  delay(500);
  digitalWrite(v, 0);
  delay(500);
  digitalWrite(v, 1);
  delay(500);
  digitalWrite(v, 0);
  delay(500);
  digitalWrite(v, 1);
  delay(500);
  digitalWrite(v, 0);
  delay(500);
  digitalWrite(v, 1);
  delay(500);
  digitalWrite(v, 0);
  delay(500);
  digitalWrite(v, 1);
  delay(500);
  digitalWrite(v, 0);
  
  digitalWrite(a, 1);
  delay(1000);
  digitalWrite(a, 0);
  digitalWrite(r, 1);
  delay(2000);
  digitalWrite(r, 0);
  
}

\end{lstlisting}
\subsection*{Diagrama Esquemático}
El diagrama esquemático es bastante parecido al de la práctica \#1, sólo se repite la misma conexión hecha en la conexión \#13, por lo tanto a partir de la conexión 11, 12 y 13 se conectará a cada una, una resistencia de 220 $\Omega$ y un LED de distinto color en serie. Cada una de estas conexiones irá a la referencia para completar el circuito eléctrico.

A continuación se muestra el diagrama esquemático:

\begin{figure}[H]
	\centering
	\includegraphics[width=\textwidth]{PR2SCH.png}
	\caption{Diagrama de las conexiones necesarias y componentes utilizados.}
\end{figure}

\section*{Conclusión}

A través de esta práctica es sencillo entender la lógica detrás del lenguaje de prorgamación Arduino, reforzando lo visto en la práctica anterior. Sin embargo, después de ver los resultados y el código, es posible concluir que es muy poco eficiente y es posible optimizarlo mucho más mediante estructuras de código más complejas como los $if\ statements$ y los bucles.
\section*{Anexos}
\subsubsection*{Hexdump}
\scriptsize
\verbatiminput{dump.txt}

%%%%%  Bib
\renewcommand\refname{References}
\printbibliography
\end{document}
