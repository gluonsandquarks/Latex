% Especificaciones del tamaño de letra, tamaño de hoja, márgenes, librerias, etc.
\documentclass[12pt, letterpaper]{article}
\usepackage[english]{babel}
\usepackage{fancyhdr}
\usepackage[utf8]{inputenc}
\usepackage[T1]{fontenc}
\usepackage{amsmath}
\usepackage{graphicx}
\usepackage{subcaption}
\usepackage[hidelinks]{hyperref}
\usepackage{url}
\usepackage{amssymb}
\usepackage{float}
\usepackage[margin=1in]{geometry}
\renewcommand{\baselinestretch}{1.5}

% Enlace Bibliografía
\usepackage{csquotes}
\usepackage[notes,backend=biber]{biblatex-chicago}
\addbibresource{referencias.bib}

% Titulo, autores, fecha.
\title{Tarea \#2: El Avión Supersónico}
\author{Carlos A. Vásquez Castañeda \and 1155057 \and Grupo 394}
\date{Febrero 13, 2020}
\pagestyle{fancy}
\fancyhf{}
\rhead{Procesos de Manufactura}
\lhead{Tarea \#2}
\rfoot{\thepage}


% Inicio del documento
\begin{document}
\maketitle

\subsection*{Materiales}
Dada la aplicación que se le dio al Concorde y a sus altas velocidades, los materiales principales que fueron utilizados son una aleación de aluminio (específicamente el Hidrominium R-58) y titanio para las partes que alcanzaban las mayores temperaturas en el avión.

\subsection*{Mano de Obra}
La construcción del Concorde es gracias a la alianza de los gobiernos y compañías inglesas y francesas. Específicamente de la British Aircraft Corporation, la Sud Aviation, Air France y la British Overseas Airways Corporations.

\subsection*{Método}
El método que se siguió, resumido en unos cuantos pasos sencillos fue (1) investigación y recopilación de datos, donde a partir de los diversos estudios generados por los fabricantes para aviones supersónicos de transporte, se llega a la conclusión de que es mucho más conveniente volar a una velocidad constante en el rango de Mach 2 que a una menor a Mach 1.5 o mayor a Mach 3. Posteriormente, mediante estos mismos datos y análisis (2) se eligen los materiales cuyas propiedades son aptas para el desempeño seguro y correcto de la aeronave. Esto considerando que, a pesar de que la opción de volar en un rango de Mach 3 también era viable, resultaba mucho más económico volar en el rango de Mach 2, por lo que estos materiales se han escogido utilizando este factor clave.

\subsection*{Máquinas}
Las máquinas utilizadas no se han específicado del todo en el documento, sin embargo, dados los fabricantes que se involucraron en el diseño y construcción del Concorde ya eran productores de aviones subsónicos, es posible sugerir que las máquinas utilizadas han sido similares a las que se han utilizado para los aviones subsónicos.

\subsection*{Medición}
Se realizaron numerosas pruebas y recopilación de datos sobre el desempeño del Concorde, y a su vez se comparó con otros aviones comerciales de la época, principalmente el tiempode vuelo, velocidad y la longitud del vuelo efectuado.

\subsection*{Medio Ambiente}
Una de las problemáticas del Concorde fue la generación de bangs supersónicos. La diferencia de presión que crean estos aviones al volar es muy grande que se les podía escuchar a grandes distancias. El gran ruido generado por el Concorde limitaba a los fabricantes a generar un avión que volara a grandes alturas, lo cual limitaba el peso de los aviones y afectava de manera negativa el punto de vista económico.

\subsection*{Recursos}
Para la fabricación de estos aviones, el principal financiamento fue asignado a los gobiernos francés e inglés, gracias a la aliación que existía entre ellos, sin embargo una mínima parte del costo se la dejaron a las empresas mencionadas anteriormente.

\section*{¿En que cambia 50 años después, el desarrollo de un avión supersónico actualizado a nuestros días?}
Los principales cambios pueden ser que la manufactura de estos aviones resulte mucho más sencilla y eficaz debido a la previa manufactura y datos obtenidos de los fabricantes con anterioridad. Las situaciones sociales y la necesidad de las personas a viajar cada vez más entre países con rapidez se presenta en vigor, lo cual podría ser una necesidad en el mercado donde un avión como el Concorde pudiese satisfacer.

La contaminación sonora es un gran factor que puede frenar la aceptación de un avión como este, y puede que esto limite el éxito de un avión similar al Concorde. La densidad de las poblaciones ha ido aumentando y las regulaciones son cada día más estrictas, así que ese campo puede ser difícil de satisfacer.
%%%%%  Bib
\renewcommand\refname{References}
\printbibliography
\end{document}
