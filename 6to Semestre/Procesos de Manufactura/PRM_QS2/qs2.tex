% Especificaciones del tamaño de letra, tamaño de hoja, márgenes, librerias, etc.
\documentclass[12pt, letterpaper]{article}
\usepackage[english]{babel}
\usepackage{fancyhdr}
\usepackage[utf8]{inputenc}
\usepackage[T1]{fontenc}
\usepackage{amsmath}
\usepackage{graphicx}
\usepackage{subcaption}
\usepackage[hidelinks]{hyperref}
\usepackage{url}
\usepackage{amssymb}
\usepackage{float}
\usepackage[margin=1in]{geometry}
\renewcommand{\baselinestretch}{1.5}

% Enlace Bibliografía
\usepackage{csquotes}
\usepackage[notes,backend=biber]{biblatex-chicago}
\addbibresource{referencias.bib}

% Titulo, autores, fecha.
\title{Cuestionario \#2: Administración de Operaciones}
\author{Carlos A. Vásquez Castañeda \and 1155057 \and Grupo 394}
\date{Marzo 16, 2020}
\pagestyle{fancy}
\fancyhf{}
\rhead{Procesos de Manufactura}
\lhead{Cuestionario \#2}
\rfoot{\thepage}


% Inicio del documento
\begin{document}
\maketitle

\begin{enumerate}
	\item ¿Cómo se organizaba la manufactura antes de la 1ª revolución industrial?

		La manufactura de productos era una interacción hombre-materia prima, en la empresa no se tenia un sistema de producción que agilizara el tiempo de manufactura. El trabajo era manual, la división de trabajo no se había definido aún. La gran parte de las personas que producían productos eran artesanos.

	\item ¿En la división del trabajo de Adam Smith cómo puede optimizarse la productividad de una empresa?

		Al momento de especializar a los trabajadores en una tarea específica se ahorraba tiempo ya que cada trabajador no debía intercambiar herramientas, ya contaba con las necesarias para realizar su trabajo, al igual que capital debido a que las herramientas que cada obrero utilizaría serían solamente las necesarias. Esto también les daba mucha más destreza en su trabajo lo cual aumentaba la eficacia y calidad del trabajo que se desarrollaba.

	\item ¿Qué permitió a Eli Whitney fabricar piezas de fusil estandarizadas más baratas e intercambiables?

		El desarrollo de máquinas-herramientas con la capacidad de crear piezas lo suficientemente idénticas para lograr ser intercambiables entre sí. Esto también facilitó la producción de las armas y estuvieron listas en un menor tiempo y precio.

	\item ¿Cuáles son las críticas que recibe la administración científica de Frederick Taylor?

		La asignación de trabajos individuales conlleva responsabilidades y tiene cirta dificultad, afectando la disciplina y moral de los empleados y no hay una gran participación por parte de los empleados.

	\item Enumere algunos de los supuestos de la teoría X y la teoría Y de Douglas Mcgregor.
		
		Esta teoría presupone que existen dos tipos de trabajadores:

		Teoría X: el trabajador es naturalmente pesimista, estático, rígido y con aversión innata al trabajo evitándolo si es posible.

		Teoría Y: trabajadores optimistas, activos, dinámicos y flexibles. Disfrutan su trabajo físico y mental.

	\item ¿De dónde parte Henry Ford para establecer el sistema de producción en línea?

		Parte de las ideas de Taylor, donde la producción del producto se divide en distintas etapas continuas donde trabajadores especializados pueden realizar pequeñas tareas que a lo largo se suman para realizar un producto complejo. Esto permitió a Henry Ford disminuir el costo de su modelo T y proporcionar automóviles económicos a los Estados Unidos.

	\item ¿Qué desventajas presenta el uso del MRP?

		El cálculo podrá ser erróneo si no se actualizan los factores y materiales constantemente. Si no existe un cálculo de capacidad requerida es probable que existan errores asimismo. Es recomendable realizarlo para productos con gran cantidad de materiales en vez de lo inverso.

	\item ¿Cómo apoya la globalización a la manufactura?

		Existe la capacidad de costos bajos mediante materiales importados, brindan oportunidad de exportación, existe la posibilidad de realizar offshore outsourcing lo cual permite expandir empresas.

	\item ¿Qué eventos encuentra que transforman la manufactura en el periodo de 2015 a 2019?

		El Internet de las Cosas (IoT), implementación de redes neuronales para optimización de procesos, datos masivos de consumidores de ciertos productos, etc.

\end{enumerate}

%%%%%  Bib
\renewcommand\refname{References}
\printbibliography
\end{document}
