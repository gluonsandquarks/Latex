% Especificaciones del tamaño de letra, tamaño de hoja, márgenes, librerias, etc.
\documentclass[12pt, letterpaper]{article}
\usepackage[english]{babel}
\usepackage{fancyhdr}
\usepackage[utf8]{inputenc}
\usepackage[T1]{fontenc}
\usepackage{amsmath}
\usepackage{graphicx}
\usepackage{subcaption}
\usepackage[hidelinks]{hyperref}
\usepackage{url}
\usepackage{amssymb}
\usepackage{float}
\usepackage[margin=1in]{geometry}
\usepackage{booktabs}
\usepackage{framed}
\usepackage{xcolor}
%\usepackage{lmodern}
\renewcommand{\baselinestretch}{1.5}

% Enlace Bibliografía
\usepackage{csquotes}
\usepackage[notes,backend=biber]{biblatex-chicago}
\addbibresource{referencias.bib}

% Titulo, autores, fecha.
\title{Tarea \#1: Cuestionario Diagnóstico}
\author{Carlos A. Vásquez Castañeda \and 1155057 \and Grupo 394}
\date{Febrero 4, 2020}
%\pagestyle{fancy}
%\fancyhf{}
%\rhead{Carlos Antonio Vásquez Castañeda}
%\lhead{Tarea \#1}
\rfoot{\thepage}
\definecolor{title}{RGB}{237, 108, 0}
\definecolor{subtitle}{RGB}{0, 117, 219}
\colorlet{shadecolor}{orange!10}

% Inicio del documento
\begin{document}
% \maketitle
%{\fontfamily{lmss}\selectfont

\begin{shaded}

\begin{flushleft}
	le 9 février
\end{flushleft}
	\begin{flushleft}
		\vspace{-0.6cm}
		\hspace{0.4cm}
		2020
	\end{flushleft}
\begin{flushright}
	\vspace{-2.5cm}
\section*{Carlos Antonio VÁSQUEZ CASTAÑEDA}
\end{flushright}
\end{shaded}

\textbf{Nom:} Vásquez Castañeda

\textbf{Prènom:} Carlos Antonio

\textbf{Date de naissance:} le 12 octobre 1998

\textbf{Lieu de naissance:} Mexicali, Basse-Californie, Mexique

\textbf{Adresse:} Mongolia Poniente \#1437. Fracc. Villanova Diamante, Mexicali, Basse-Californie, C.P. 21307

\textbf{Téléphone:} +52 (686) 197-4145

\textbf{E-mail:} workflow4granted@gmail.com

\textbf{Âge:} 21 ans

\subsection*{\textcolor{title}{O B J E C T I F S}}
Mon objectif principal dans la vie est d'éviter la mediocrité. Je me demande toujours qu'il faut pour réussir et devenir une meilleur personne. Étudier et collaborer avec d'autres pour développer des projets fait partie de mes objectifs.

Terminer mes études universitaires, étudier l'enérgetique et la propulsion dans les systèmes aérospatiaux, obtenir une maîtrise et un doctorat. Ces sont mes objectifs dans la vie actuellement.



\begin{shaded}
	\vspace{-0.3cm}
	\subsection*{\textcolor{title}{F O R M A T I O N\ \ E T \ \ É T U D E S}}

	\textbf{\textcolor{subtitle}{Centro de Bachillerato Tecnológico Industrial y de Servicios \#21 (2014 --- 2017)}}
	
	\textbf{Technicien de Laboratoire Clinique}

	Certificat d'études (Diplôme du lycée) / Moyenne: 9,9/10

	\noindent \textbf{\textcolor{subtitle}{Université Autonome de L'État de  Basse-Californie --- Ingénierie Aérospatial, Sixième Semestre (2017 --- Présent)}}

	Actuellement une moyenne de 96,33/100

	\noindent \textbf{\textcolor{subtitle}{Stage Intensif au CAVILAM, Alliance Française de Vichy (Novembre 2019 --- Décembre 2019)}}

	Niveau actuel: B1+

	Certificat d'études

\end{shaded}


\subsection*{\textcolor{title}{L A N G U E S}}

\begin{itemize}
	\item Espagnol: Langue maternelle.
	\item Anglais: Niveau B2.
	\item Français: Niveau B1.
\end{itemize}


\begin{shaded}
	\vspace{-0.3cm}
	\subsection*{\textcolor{title}{C O M P É T E N C E S}}
	
	\begin{itemize}
		\item Projets y conaissance en programmation en langage C, Arduino, JavaScript, Python, p5.js, MATLAB, LaTeX, NI LabView.
		\item Développeur Front-End de sites Internet. (via des outils comme HTML, CSS, JavaScript, Gatsby, React, Node)
		\item Conaissance des systèmes d'exploitation basés sur Linux (Ubuntu, Arch Linux, Kali Linux) et utilisation de logiciels libres et open source (FOSS).
		\item Expérience en modélisation, conception et assemblage via le logiciel CATIA V5R20 de la société Dassault Systèmes.
		\item Conception graphique avec des logiciels de la société Adobe (Photoshop, Illustrator) et Gimp.
		\item Création de contenu multimédia avec Adobe Premiere, Adobe Photoshop, Adobe After Effects et DaVinci Resolve (courts métrages, effets spéciaux, animations)
		\item Conaissance de Microsoft Office (Word, Excel, PowerPoint).
	\end{itemize}
\end{shaded}


\subsection*{\textcolor{title}{E X P E R I E N C E S\ \ D E\ \ T R A V A I L}}

	\noindent \textbf{\textcolor{subtitle}{Reactor Indie (2015 --- 2016)}}
	
	\textbf{Designer Principal}

	Creation de contenu multimédia (annonces, vidéos spectaculaires)

	\noindent \textbf{\textcolor{subtitle}{Développeur  Web (2019 --- Présent)}}
	
	\textbf{Développeur Front-End}

	Conception des pages principales, foncionnalité de base, synchronisation d'APIs et fonctions diverses.

%%%%%  Bib
\renewcommand\refname{References}
\printbibliography
%}
\end{document}
