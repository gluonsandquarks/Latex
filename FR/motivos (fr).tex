% Especificaciones del tamaño de letra, tamaño de hoja, márgenes, librerias, etc.
\documentclass[12pt, letterpaper]{article}
\usepackage[english]{babel}
\usepackage{fancyhdr}
\usepackage[utf8]{inputenc}
\usepackage[T1]{fontenc}
\usepackage{amsmath}
\usepackage{graphicx}
\usepackage{subcaption}
\usepackage[hidelinks]{hyperref}
\usepackage{url}
\usepackage{amssymb}
\usepackage{float}
\usepackage[margin=1in]{geometry}
\renewcommand{\baselinestretch}{1.5}

% Enlace Bibliografía
\usepackage{csquotes}
\usepackage[notes,backend=biber]{biblatex-chicago}
\addbibresource{referencias.bib}

% Titulo, autores, fecha.
\title{Tarea \#1: Cuestionario Diagnóstico}
\author{Carlos A. Vásquez Castañeda \and 1155057 \and Grupo 394}
\date{Febrero 4, 2020}
%\pagestyle{fancy}
%\fancyhf{}
%\rhead{Procesos de Manufactura}
%\lhead{Tarea \#1}
\rfoot{\thepage}


% Inicio del documento
\begin{document}
% \maketitle
\hspace*{\fill} le 9 février 2020

\hspace*{\fill} Mexicali, Basse-Californie, Mexique

\begin{center}
	Lettre de Motifs
\end{center}

À qui de droit:

Je m'appelle Carlos Antonio VÁSQUEZ CASTAÑEDA, actuellement je suis étudiant de sixième semestre de l'ingénierie aérospatial dans la faculté d'ingénierie, en l'Université Autonome de Basse-Californie (UABC) à Mexicali.

Je me suis demandé toute ma vie ce qu'il fallait pour devenir une meilleur personne, éviter la médiocrité et avoir l'esprit tranquille avec moi-même. Ces questions m'ont rendu curieux, elles m'ont donné le besoin de comprendre le monde autour de moi et le besoin de faire quelque chose avec le connaissance derrière moi. Grâce à ces motivations je continue mes études même en dehors de l'école, je fait des projets personnels, j'étude seul (autodidacte) et j'ai une bonne histoire academique.

Depuis mon enfance j'ai compris que la France est un pays où l'innovation, la science et l'art sont très important, cela m'a beaucoup fait l'admirer. Depuis j'étais adolescent j'ai eu des contacts avec la langue et culture française. L'opportunité de faire un échange en France me donne de l'énergie pour chercher le savoir que j'avais cherché toute ma vie.

Je suis sûr que cet échange m'apportera tout le nécessaire pour me devenirun ingenieur competent. Je voudrais apprendre avec les méthodes françaises et pouvoir enseigner ce que je trouverai en France au pays qui m'a vu grandir. Ces sont mes objectifs principals. Au Mexique il y a beaucoup des opportunités et une de mes rêves est pouvoir faire quelque chose pour le lieu où je suis né.

Je crois que c'est possible s'il y du travail d'équipe, et je suis sûr que je peux travailler pour obtenir les meilleurs résultats possibles.

Merci pour votre temps et considération.

\hspace*{\fill} Cordialement,

\hspace*{\fill} Signature: \noindent\rule{6cm}{0.4pt}

\hspace*{\fill} VÁSQUEZ CASTAÑEDA Carlos Antonio
%%%%%  Bib
\renewcommand\refname{References}
\printbibliography
\end{document}
