% Especificaciones del tamaño de letra, tamaño de hoja, márgenes, librerias, etc.
\documentclass[12pt, letterpaper]{article}
\usepackage[english]{babel}
\usepackage{fancyhdr}
\usepackage[utf8]{inputenc}
\usepackage[T1]{fontenc}
\usepackage{amsmath}
\usepackage{graphicx}
\usepackage{subcaption}
\usepackage[hidelinks]{hyperref}
\usepackage{url}
\usepackage{amssymb}
\usepackage{float}
\usepackage[margin=1in]{geometry}
\renewcommand{\baselinestretch}{1.5}
\setlength\parindent{0pt}
% Enlace Bibliografía
\usepackage{csquotes}
\usepackage[notes,backend=biber]{biblatex-chicago}
\addbibresource{referencias.bib}

% Titulo, autores, fecha.
\title{Tarea \#1: Densidad, Peso Específico y Gravedad Específica}
\author{Carlos Vásquez \and 1155057}
\date{Agosto 19, 2019}
\pagestyle{fancy}
\fancyhf{}
\rhead{Carlos Vásquez}
\lhead{Tarea \#1 --- Dinámica de Fluidos}
\rfoot{\thepage}


% Inicio del documento
\begin{document}
\maketitle

\textbf{Problema 1.81M} La gravedad específica del benceno es de 0.876. Calcule su peso específico y su densidad, en unidades del SI. \\
\textbf{Datos} \hspace{5cm} \textbf{Fórmulas} \\
$SG_{C_6H_6} = 0.876$ \hspace{3.3cm} $SG = \frac{\rho}{\rho_{H_2O}} = \frac{\gamma}{\gamma_{H_2O}}$\\
$\rho_{H_2O\ @\ 4\ ºC} = 1000 \frac{kg}{m^3}$\\
$\gamma_{H_2O\ @\ 4\ ºC} = 9.81 \frac{kN}{m^3}$


\textbf{Solución} \\
Dada la fórmula de la gravedad específica, podemos obtener el peso específico y la densidad del benceno si asumimos que la temperatura del agua de referencia se encuentra a 4 ºC.

\begin{equation*}
	\begin{split}
		SG &= \frac{\rho_{C_6H_6}}{\rho_{H_2O\ @\ 4\ ºC}}\\
		0.876 = \frac{\rho_{C_6H_6}}{1000\frac{kg}{m^3}} \Rightarrow \rho_{C_6H_6} &= (0.876)(1000 \frac{kg}{m^3})
	\end{split}
\end{equation*}

\begin{equation}
	\boxed{
		\rho_{C_6H_6} = 876 \frac{kg}{m^3}}
\end{equation}

\begin{equation*}
	\begin{split}
		\gamma_{C_6H_6} &= (SG)(\gamma_{H_2O\ @\ 4\ ºC})\\
		\gamma_{C_6H_6} &= (0.876)(9.81 \frac{kN}{m^3})
	\end{split}
\end{equation*}

\begin{equation}
	\boxed{
		\gamma_{C_6H_6} = 8.59356 \frac{kN}{m^3} \approx 8.6 \frac{kN}{m^3}}
\end{equation}


\textbf{Problema 1.83M} Cierto aceite medio de lubricación tiene un peso específico de 8.860 $kN/m^3$ a 5 ºC, y 8.483 $kN/m^3$ a 50 ºC. Calcule su gravedad específica en cada temperatura. \\
\textbf{Datos} \hspace{5cm} \textbf{Fórmulas} \\
$\gamma_{aceite\ @\ 5\ ºC} = 8.860 \frac{kN}{m^3}$ \hspace{1.8cm} $SG = \frac{\gamma}{\gamma_{H_2O}}$\\
$\gamma_{aceite\ @\ 50\ ºC} = 8.483 \frac{kN}{m^3}$\\
$\gamma_{H_2O\ @\ 5\ ºC} = 9.81 \frac{kN}{m^3}$\\
$\gamma_{H_2O\ @\ 50\ ºC} = 9.69 \frac{kN}{m^3}$


\textbf{Solución} \\
Con la fórmula de la gravedad específica y los datos proporcionados en el apéndice A del libro es posible calcular la gravedad específica a las distintas temperaturas requeridas.
\begin{equation*}
	\begin{split}
		SG_{@\ 5\ ºC} &= \frac{\gamma_{aceite\ @\ 5\ ºC}}{\gamma_{H_2O\ @\ 5\ ºC}}\\
		&= \frac{8.860 \frac{kN}{m^3}}{9.81 \frac{kN}{m^3}}
	\end{split}
\end{equation*}

\begin{equation}
	\boxed{SG_{@\ 5\ ºC} \approx 0.9032}
\end{equation}

\begin{equation*}
	\begin{split}
		SG_{@\ 50\ ºC} &= \frac{\gamma_{aceite\ @\ 50\ ºC}}{\gamma_{H_2O\ @\ 5\ ºC}}\\
		&= \frac{8.483 \frac{kN}{m^3}}{9.69 \frac{kN}{m^3}}
	\end{split}
\end{equation*}

\begin{equation}
	\boxed{SG_{@\ 50\ ºC} \approx 0.8754}
\end{equation}

\textbf{Problema 1.85M} Una lata ciíndrica de 150 mm de diámetro contiene 100 mm de aceite combustible. El aceite tiene una masa de  1.56 kg. Calcule su densidad, peso específico y gravedad específica. \\
\textbf{Datos} \hspace{5cm} \textbf{Fórmulas} \\
$\varnothing_{lata} = 150 mm$ \hspace{3.4cm} $V = h\pi r^2$\\
$h = 100 mm$ \hspace{4cm} $\rho = \frac{m}{V}$\\
$m = 1.56 kg$ \hspace{4cm} $\gamma = \frac{W}{V}$\\
$\rho_{H_2O\ @\ 4\ ºC} = 1000 \frac{kg}{m^3} $ \hspace{2cm} $SG = \frac{\rho}{\rho_{H_2O}}$\\
$g = 9.81 \frac{m}{s^2}$ \hspace{4cm} $W = mg$

\textbf{Solución} \\

Fuuuuuuuckkk this imma make it handwritten. AYeeeeeeeeeeeeeeeeee
%%%%%  Bib
\renewcommand\refname{Referencias}
\printbibliography
\end{document}
