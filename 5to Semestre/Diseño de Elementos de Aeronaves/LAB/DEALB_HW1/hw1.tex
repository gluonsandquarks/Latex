% Especificaciones del tamaño de letra, tamaño de hoja, márgenes, librerias, etc.
\documentclass[12pt, letterpaper]{article}
\usepackage[english]{babel}
\usepackage{fancyhdr}
\usepackage[utf8]{inputenc}
\usepackage[T1]{fontenc}
\usepackage{amsmath}
\usepackage{graphicx}
\usepackage{subcaption}
\usepackage[hidelinks]{hyperref}
\usepackage{url}
\usepackage{amssymb}
\usepackage{float}
\usepackage[margin=1in]{geometry}
\renewcommand{\baselinestretch}{1.5}

% Enlace Bibliografía
\usepackage{csquotes}
\usepackage[notes,backend=biber]{biblatex-chicago}
\addbibresource{referencias.bib}

% Titulo, autores, fecha.
\title{Tarea \#1: Análisis de Torsión para Pieza Sólida y Hueca}
\author{Carlos A. Vásquez Castañeda \and 1155057 \and Grupo 392}
\date{Agosto 21, 2019}
\pagestyle{fancy}
\fancyhf{}
\rhead{Diseño de Elementos de Aeronaves}
\lhead{Tarea \#1}
\rfoot{\thepage}


% Inicio del documento
\begin{document}
\maketitle
\section*{Introducción}
Los motores han sido de gran estudio a lo largo de la ingeniería, éstos operan bajo conceptos sencillos y su gran versatilidad ha permitido que distintas partes de la tecnología hallan avanzado en las últimas décadas. No todos aquellos están creados de la misma manera ni con el mismo fin, pero existe una variedad tan extensa que es difícil imaginar el mundo moderno sin éstos.

En la industria aeroespacial los distintos motores utilizados varían dependiendo del fin, pero si algo es seguro es que éstos deben ser muy fiables y probados para asegurar su correcto funcionamiento. A continuación analizaremos los motores más utilizados en la industria.

\subsection*{Turbofan}

\subsection*{Turboprup}

\subsection*{Turbojet}

\subsection*{Turboshaft}

\subsection*{RAT}

\subsection*{APU}
%%%%%  Bib
\renewcommand\refname{Referencias}
\printbibliography
\end{document}
